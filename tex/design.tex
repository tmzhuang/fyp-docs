\chapter{Design and Implementation}
\label{chap:design}

\section{Design}
The full design of this system is show in Figure~\ref{fig:design_full}. Since the system is not critical, there are no componments dealing with the redundancy of the system. Many of the classes were extracted from the requirements use the verb-noun method.
\begin{center}
    \includegraphics[width=.8\textwidth]{design_full.png}
    \captionof{figure}{Full class diagram}
    \label{fig:design_full}
\end{center}

Figure~\ref{fig:design-p1} shows several classes required for achieving the logging requirement.
\begin{center}
    \includegraphics[width=.9\textwidth]{design_p1.png}
    \captionof{figure}{Logging classes}
    \label{fig:design-p1}
\end{center}


Figure~\ref{fig:design-p2} shows several classes required for achieving the graphing requirement.
\begin{center}
    \includegraphics[width=.9\textwidth]{design_p2.png}
    \captionof{figure}{Graphing classes}
    \label{fig:design-p2}
\end{center}

\section{Implementation}
Implementation was achieved using the Ruby language. This was a fit since way the program is hooked into target application uses a Ruby class, TracePoint. The souce code is available for viewing at the project home page (see Appendix~\ref{chap:appendix}).Here is an excerpt from {\lstinline |grapher.rb|}:
\begin{lstlisting}[
  language=Ruby,
  breaklines=true
]
    def create_clusters(method)
      cluster_options = {tooltip: " "}
      if method.types.include?(:service)
        cluster_options[:bgcolor] = :azure3
      elsif method.types.include?(:framework)
        cluster_options[:bgcolor] = :brown
      end

      # Iterate over ancestors (eg. Minitest::Unit yields Minitest, # Unit)
      # This variable is used to reference the immediate parent of the 
      # current graph. Initializes to the root; updates on each iteration.
      prev_graph = @graph 
      method.ancestors.each_with_index do |klass, i|
        # Prepending "cluster_" is mandatory according to ruby-graphviz API
        graph_id = "cluster_" + method.ancestors[0..i].join("_")
        # add_graph returns an existing graph if it exists
        subgraph = prev_graph.add_graph(graph_id, cluster_options)
        subgraph[:label] = klass
        # Update parent for next iteration
        prev_graph = subgraph
      end
      prev_graph
    end
\end{lstlisting}
