\appendix

\chapter{Project Repository}
\label{chap:repo}
The project was kept under version control using git through GitHub. The repository can be found at https://github.com/tmzhuang/dependency\_grapher. Additionally, the tool was tool packaged in a Ruby Gem (a common way of sharing Ruby code) and released on RubyGems.org. Instructions on installation and operation of the tool can be found in the readme of the repository. It is included here for the convenience of the reader.

\section{Installations}
Add this line to your application's Gemfile:
\begin{lstlisting}[
  language=Ruby,
  breaklines=true
]
gem 'dependency_grapher'
\end{lstlisting}
And then run {\lstinline|bundle install|}. 

\section{Usage}
In your test helpers file {\lstinline|test/test_helper|}, include {\lstinline|DependencyGrapher::TestHelpers|} to your tests as shown below:
\begin{lstlisting}[
  language=Ruby,
  breaklines=true
]
class ActionController::TestCase
  include DependencyGrapher::TestHelpers
  #include DependencyGrapher::TestHelpers
end

class ActiveSupport::TestCase
  include DependencyGrapher::TestHelpers
  # Setup all fixtures in test/fixtures/*.yml for all tests in alphabetical order.
  fixtures :all
end
\end{lstlisting}
DependencyGrapher requires you first to run your unit tests to produce set of dependencies (outputted to your project folder as dependencies.yml). The parts of your system that are not touched by the tests will not be shown in the graph.

After the tests have complete, run {\lstinline|rake dep:graph|} to produce a SVG graph. You may specify the name format of the graph output by using {\lstinline|rake dep:graph[name.format]|}. For example, {\lstinline|rake dep:graph[graph.dot]|} produces a DOT file of the graph. Acceptable formats can be found on the GraphViz site.
