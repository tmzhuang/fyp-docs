\chapter{Introduction}
Development tools are an important part of software development. The end goal of these tools is to reduce the number of faults in the system and thereby improving the quality of the code. Examples range from integrated development environments (IDE) that provide syntax checking, to debuggers that examine the state of the system during runtime. The aim of this project is to create a development tool that detects violations of design patterns implemented in software systems. Specifically, the tool is designed for developers of Ruby on Rails applications. To achieve this end, the tool captures call dependencies within the target system, and then output a call graph with violations highlighted. 

Ruby on Rails is a web development framework written in the Ruby language. It assists in producing web applications following a RESTful API and using a MVC pattern for its user interface. Setting up a new project is as simple as  {\lstinline[breaklines=true] |rails new project_name|}. This creates an application skeleton with several folders. Figure~\ref{fig:project-tree} shows an sample of the {\lstinline |app/|} folder created by this command.

\begin{center}
    \includegraphics[width=.5\textwidth]{rails_project_tree.png}
    \captionof{figure}{Folder structure of new Rails project}
    \label{fig:project-tree}
\end{center}
Most of the code is organized into the {\lstinline |app/|} folder; these are further subdivided into the {\lstinline |models/|}, {\lstinline |views/|}, and {\lstinline |controllers/|} folders. Users can generate new sets of controllers, models, and views using the {\lstinline |rails generate scaffold name|} command. There are many other such tools (implemented with Rake tasks, a Ruby-flavoured make), that deal with the creation of skeleton code.

In addition to this basic structure, Rails comes with its own server and database (WEBrick and sqlite by default, respectively). Figure~\ref{fig:mvc} depicts the MVC model of Rails. Rails provides tools to help handle all aspects of this model. However, this mostly deals with user interface; where should code that does not deal with the interface, such as business logic, go? 
\begin{center}
    \includegraphics[width=.8\textwidth]{mvc_detailed.png}
    \captionof{figure}{MVC model of Rails \cite{rails-tutorial}}
    \label{fig:mvc}
\end{center}
In this particular case, there is an interest to follow domain-driven design. In model-driven development, domain objects are classified into several classes: entities, object values, aggregates, domain events. Furthermore, there are factory and repository objects which handle the creation and storing of domain objects respectively. Operations which do not belong to any domain object are put into service objects~\cite{ddd-source}. More specifically, in order to separate business logic from logic that handles database or UI, one possible solution is to encapsulate it in a service object. The service objects handle any interaction that might be necessary between the system and external actors (such as interfacing with an external framework). In this way, the system becomes more modular (higher cohesion, looser coupling). This practice may also make the system more testable as this simplifies methods and the classes that contain them. 

The project addresses a common problem for developers working on complex systems. It can become difficult to keep track of the service objects that exist for certain frameworks. Additionally, there may be legacy code which did not adhere to the previously discussed use of service objects for system interactions. The goal of this project is to create a development tool to better visualize the complex web application systems. The tool highlights where there exists a service object intended for interacting with an external framework, but an existing class directly makes calls to that same framework (Figure~\ref{fig:violation}).
\begin{center}
    \includegraphics[width=.8\textwidth]{violation_example.png}
    \captionof{figure}{Solution example}
    \label{fig:violation}
\end{center}

The proposed solution is to implement the described tool which produces a call graph of a given Rails application. The call graph highlights the parts of the system which make "violating" calls. 

Although the initial scope of the project was to develop a tool that could detect many types of violations, the completed tool detects only violations regarding misuse of service objects. The tool developed is able to produce a log of dependencies of a target Rails application and generate a call graph as shown in Figure~\ref{fig:result}. This project employed an iterative development approach. However, due to time limitations, only one iteration was completed. Thus, there may be more requirements to be refined. The solution produced only matches the requirements elicited in the first iteration.
\begin{center}
  \includegraphics[trim={0 0 60cm 0}, clip, width=.95\textwidth]{result.png}
    \captionof{figure}{Final result}
    \label{fig:result}
\end{center}

The upcoming chapters illustrates how the final solution was achieved. In Chapter~\ref{chap:project}, this report discusses the logistics of this project from an engineering perspective. Following, Chapter~\ref{chap:requirement} discusses requirement elicitation and analysis. Chapter~\ref{chap:design} discusses the system design and implementation. And lastly, Chapter~\ref{chap:verification} discusses system verification and validation. There is no discussion on operation and maintenance because this project did not reach deployment stage.
