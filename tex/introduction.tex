\chapter{Introduction}

Ruby on Rails is a web development framework written in the Ruby language. It assists in producing web applications following a RESTful API and using a MVC pattern for its user interface. Setting up a new project is as simple as  {\lstinline |rails new project_name|}. This creates an application skeleton with several folders. Figure~\ref{fig:project-tree} shows an example of the {\lstinline |app/|} folder created by this command.

\begin{center}
    \includegraphics[width=.5\textwidth]{rails_project_tree.png}
    \captionof{figure}{Folder structure of new rails project}
    \label{fig:project-tree}
\end{center}
Most of the code is organized into the {\lstinline |app/|} folder, further subdivided into the {\lstinline |models/|}, {\lstinline |views/|}, and {\lstinline |controllers/|} folders. Users can generate new set of controllers, models, and views using the {\lstinline |rails generate scaffold name|} command. There are many other such tools (implemented with Rake tasks, a Ruby flavoured make), dealing with the creation of such skeleton code.

In addition to this basic structure, Rails comes with its own server and database (WEBrick and sqlite by default, respectively). Figure~\ref{fig:mvc} depicts the MVC model of Rails. Rails provides tools to help handle all aspects of this model. However, this mostly deals with user interface; where should code that does not deal with the interface, such as business logic, go? 
\begin{center}
    \includegraphics[width=.8\textwidth]{mvc_detailed.png}
    \captionof{figure}{MVC model of Rails \cite{rails-tutorial}}
    \label{fig:mvc}
\end{center}
In this particular case, there is an interest to follow domain-driven design. In model-driven development, domain objects are classified into several classes: entities, object values, aggregates, domain events. Furthermore, there are factory and repository objects which handle the creation and storing of domain objects respectively. Operations which do not belong to any domain object are put into service objects~\cite{ddd-source}. More specifically, in order to separate business logic from logic that handles database or UI, one possible solution is to encapsulate it in a service objects. The service objects handle any interaction that might be necessary between the system and external actors (such as interfacing with an external framework). In this way, the system becomes more modular (higher cohesion, looser coupling). This practice could also make the system more testable as this generally simplifies methods and the classes that contain them. 

The problem of addressed by this project is that for developers working on complex systems, it can become to keep track of the service objects that exist for certain frameworks. Additionally, there could be legacy code that exists which did not adhere to the previously discussed use of service service objects for system interactions. The goal of this project is to provide developers a tool to visualize the systems they are implementing. The tool should make it clear to the user in cases where there exists a service object intended for interacting with an external framework, but another class directly makes calls to that same framework (Figure~\ref{fig:violation}).
\begin{center}
    \includegraphics[width=.8\textwidth]{violation_example.png}
    \captionof{figure}{Solution example}
    \label{fig:violation}
\end{center}

The proposed solution is to implement the described tool which can produce a call graph of a given Rails application. The call graph will highlight the parts of the system which make "violating" calls.

The final result of the project matches those of the requirements. The tool developed is able to produce a log of dependencies of a target Rails application and generate a call graph like the one shown in Figure~\ref{fig:result}. This project tried to take an iterative development approach. However, due to time factors, only one iteration was completed. Thus, there may be more requirements to be refined. The solution produced only matches the requirements elicited in the first iteration.
\begin{center}
  \includegraphics[trim={0 0 60cm 0}, clip, width=.95\textwidth]{result.png}
    \captionof{figure}{Final result}
    \label{fig:result}
\end{center}

The upcoming chapters will illustrate how the final solution was achieved. In Chapter~\ref{chap:project}, this report will discuss the logistics of this project from an Engineering perspective. Following, Chapter~\ref{chap:requirement} will discuss requirement elicitation and analysis. Chapter~\ref{chap:design} will discuss the system design and implementation. And lastly, Chapter~\ref{chap:verification} will discuss system verification and validation. This project did not have a chance to reach deployment stage, so there will be no discussion on operation and maintenance. 
