\chapter{Verification and Validation}
\label{chap:verification}
Verification was automated in the system were possible. Due to the relatively small size of the system, a hundred percent edge coverage is achieved for tests dealing with internal logic. Tests dealing with system interaction or producing output was not automated. Although automating these processes would have increased efficiency and effectiveness, there was simply not enough time to implement such tests. Instead, tests involving the comparison between expected output of the system with actual output was done manually.

Several testing frameworks were used to assist in this verification. The unit-testing framework used was Minitest, which is included by default with Ruby. In addition Mocha was used for stubbing. It was necessary to stub certain classes throughout the development of the system as a TDD approach was taken and it was common for tests to call classes which had not yet been implemented. Additionally, Mocha was very useful in creating test data. Guard was a gem which provided continuous testing. It provides a binary which monitors specified folders and runs the associated unit tests when files are changed. For example, saving to the logger.rb file or the test\_logger.rb file runs the test in TestLogger. Such a tool helps ensure that changes to the system do not break old functionality.
