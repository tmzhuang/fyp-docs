\chapter{Background Literature Review}

%=================================================================================

\section{A Few \LaTeX\ Examples}

You can reference great written works like this \cite{ABC} or
others like this \cite{XYZ}.

%=================================================================================

\section {Mathematical Equations}
Simple equations, like $x^y$ or $x_n = \sqrt{a + b}$ can be typeset right
in the text line by enclosing them in a pair of single dollar sign symbols.
Don't forget that if you want a real dollar sign in your text, like \$2000,
you have to use the \verb+\$+ command.

An example equation is
\begin{equation} \label{E:myfirst}
	A = B
\end{equation}
This was equation \eqref{E:myfirst}.

A more complicated equation should be typeset in {\em displayed math\/} mode,
like this:
\[
	z \left( 1 \ +\ \sqrt{\omega_{i+1} + \zeta -\frac{x+1}{\Theta +1} y + 1} 
	\ \right)
	\ \ \ =\ \ \ 1
\]
The "equation" environment displays your equations, and automatically
numbers them consecutively within your document, like this:
\begin{equation}
	\left[
		{\bf X} + {\rm a} \ \geq\ 
		\underline{\hat a} \sum_i^N \lim_{x \rightarrow k} \delta C
	\right]
\end{equation}


Other environments exist, like the "align" environment.  For instance,
the {\em unitary} Fourier transform pair is given as
\begin{align}
	X(j\Omega) &= \frac{1}{\sqrt{2\pi}} \int_{-\infty}^{\infty} x(t) e^{-j\Omega t} dt\\
	x(t) &= \frac{1}{\sqrt{2\pi}} \int_{-\infty}^{\infty} X(j\Omega) e^{j\Omega t} d\Omega
\end{align}

Here is a matrix:
$$
\left[ \begin{matrix} 1 & 2 \\ 3 & 4 \end{matrix} \right]
$$


Possible useful text environments could include the following.

\begin{lemma} 
	This is a lemma.
\end{lemma}

\begin{theorem} \label{T:my1}
	This is a theorem.
\end{theorem}


\begin{proof}
	This is the proof of Theorem~\ref{T:my1}.
\end{proof}


\begin{definition}
	This is a definition.
\end{definition}

\begin{notation}
	This is some notation.
\end{notation}

%=================================================================================

\section{Example Figure}

%An example figure grabbed from the Carleton University webpage (\url{http://www.carleton.ca}) is shown in Fig.~\ref{fig-culogo}.
%\begin{figure}[hbt]
	%\begin{center}
		%\includegraphics[width=\textwidth,height=\textheight,keepaspectratio]{mvc_detailed.png}
	%\end{center}
	%\caption{Carleton University logo.}
	%\label{fig-culogo}
%\end{figure}

%%%%%%%%%%%%%%%%%%%%%%%%%%%%%%%%%%%%%%%%%%%%%%%%%%%%%%%%%%%%%%%%%%%%%%%%%%%%%%%%%%

%\chapter{Your Implementation}

%%%%%%%%%%%%%%%%%%%%%%%%%%%%%%%%%%%%%%%%%%%%%%%%%%%%%%%%%%%%%%%%%%%%%%%%%%%%%%%%%%

%\chapter{Experimental Results}

%%%%%%%%%%%%%%%%%%%%%%%%%%%%%%%%%%%%%%%%%%%%%%%%%%%%%%%%%%%%%%%%%%%%%%%%%%%%%%%%%%

